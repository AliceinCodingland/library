\documentclass[xcolor=dvipsnames,handout]{beamer}
\usecolortheme[named=Brown]{structure}
\usetheme{PaloAlto}
\useoutertheme{infolines}

\usepackage{url}

\newcommand\lingo[1]{\textit{#1}}
\newcommand\Dash{\unskip\thinspace\textemdash\thinspace\ignorespaces\-}
\newcommand\prog[1]{\textit{#1}}
\newcommand\meta[1]{$\langle${\sffamily\itshape #1}$\rangle$}
\newcommand\bash[1]{\fcolorbox{gray}{green!15}{{\ttfamily\footnotesize #1}}}
\newcommand\Asana[1]{\hfill \textcolor{gray!50}{\footnotesize Asana's `#1'}}

\author{Library Team}
\title{A Quick Introduction to GitHub}
\date{30 January 2014}

\begin{document}
\maketitle

\section{Name of the Game}
\begin{frame}
  \note{How many people already know about git?}
  \begin{center}
    GitHub is for \texttt{git}!
  \end{center}
\end{frame}

\subsection{GitHub's Definitions}
\begin{frame}
  \frametitle{WhatHub?}
  \begin{description}[<+(1)->]
  \item[git]
    an extremely fast, efficient, distributed version control system
    ideal for the collaborative development of software
  \item[GitHub]
    the best way to collaborate with others.
    Fork, send pull requests, and manage
    all of your \emph{public} and \emph{private} git repositories
  \end{description}
\end{frame}

\subsection{My Definitions}
\begin{frame}
  \frametitle{GitHub is for \texttt{git}!}
  \begin{description}[<+(1)->]
  \item[git]
    an extraordinary little command-line tool that
    keeps a history of the project over time
  \item[GitHub]
    a \emph{complement} to git, providing things such as
    an advanced, commit-integrated issue tracking system,
    a wiki, and access to various metrics
  \end{description}
\end{frame}

\section{Using the Tools}
\begin{frame}
  \frametitle{Using the Tools}
  \begin{center}
    Git and GitHub are \emph{complementary} to each other \Dash they
    have different purposes and different tools to work with them.
  \end{center}
\end{frame}

\subsection{Git}
\begin{frame}
  \frametitle{Major Concepts}
  \begin{description}[<+(1)->]
  \item[stage] add specific local changes to a \lingo{staging area},
    where they can be reviewed
  \item[commit] finalize local changes, assigning it an identifier and adding it to the log
  \item[tag] bookmark a state of the repository for later reference (e.\,g.~\texttt{milestone1})
  \item[push] make your local commits available to everyone
  \item[pull] update your local copy to mirror GitHub
  \item[branch] create your own `private working copy' (to add a feature, resolve an issue, etc.)
    \note{You should always create a tracking branch on GitHub.
      The `how' of this will vary from client to client,
        but you can always run \bash{git push origin my-local-branch-name -u}.}
  \item[merge] \lingo{pull} changes from one branch into another
  \end{description}
\end{frame}

\subsubsection{Clients}
\begin{frame}
  \frametitle{Clients}
  \begin{itemize}[<+(1)->]
  \item Since git is a command-line tool, many people use \lingo{clients}
  \item \alert{GitHub is \emph{not} a git client} \Dash it is a \emph{complement} to git
  \item Recommended clients include \prog{Magit}, \prog{SmartGit}, \prog{SourceTree},
    and platform-dependent clients released by GitHub
  \item All of the above except \prog{Magit} come with GitHub support baked-in
  \end{itemize}
\end{frame}

\subsubsection{Getting Started}
\begin{frame}
  \frametitle{Using Git}
  \begin{itemize}[<+(1)->]
  \item Download from \url{http://www.git-scm.org}
  \item Tell git your name and email:
    \begin{description}[<3->]
      \setlength\itemindent{-3em}
    \item[name]  \bash{git config --global user.name \ "Sean Allred"}
    \item[email] \bash{git config --global user.email  "myemail@..."}
    \end{description}
  \item Run \bash{git init} or \bash{git clone \meta{remote}}
  \end{itemize}
\end{frame}

\subsection{GitHub}
\begin{frame}
  \frametitle{Major Concepts}
  \begin{description}[<+(1)->]
  \item[issue] bug reports and feature requests \Asana{task}
  \item[milestone] major, measureable goal \Asana{task section\slash?}
  \item[pull request] discussion on a set of changes \mbox{(think~\lingo{pull} and~\lingo{merge})}
  \item[comment] discussion on a file or particular commit
  \item[pulse] summary of recent activity in the repository
  \end{description}
\end{frame}

\subsubsection{Annoyances}
\begin{frame}
  \frametitle{Annoyances}
  \begin{itemize}[<+(1)->]
  \item \texttt{email-hurricane-mode} enabled by default \Dash\\ use `Account Settings' to control
  \item not intuitive if you don't already understand git
  \end{itemize}
\end{frame}

\subsubsection{Getting Started}
\begin{frame}
  \frametitle{Getting Started}
  \begin{itemize}[<+(1)->]
  \item Make an account on \url{http://www.github.com}
  \item Send your username along to Alexia to get set up
  \item Attach your git client to your GitHub account
  \item Start participating!
  \end{itemize}
\end{frame}

\section{Further Resources}
\begin{frame}
  \frametitle{Further Resources}
  \begin{itemize}
  \item \url {http://try.github.io}
  \item \url {http://git-scm.org/book}
  \item \url {http://git-scm.com/downloads/guis}
  \item \url {http://rogerdudler.github.io/git-guide}
  \item \url {http://training.github.com/resources/videos}
  \item \url {https://speakerdeck.com/matthewmccullough/intro-to-git-and-github}
  \item \url {https://speakerdeck.com/matthewmccullough/git-and-github-workflows}
  \end{itemize}
\end{frame}
\end{document}

%%% Local Variables: 
%%% mode: latex
%%% TeX-master: t
%%% TeX-PDF-mode: t 
%%% End: 
